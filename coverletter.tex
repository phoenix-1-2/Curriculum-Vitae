%!TEX TS-program = xelatex
%!TEX encoding = UTF-8 Unicode
% Awesome CV LaTeX Template for Cover Letter
%
% This template has been downloaded from:
% https://github.com/posquit0/Awesome-CV
%
% Authors:
% Claud D. Park <posquit0.bj@gmail.com>
% Lars Richter <mail@ayeks.de>
%
% Template license:
% CC BY-SA 4.0 (https://creativecommons.org/licenses/by-sa/4.0/)
%


%-------------------------------------------------------------------------------
% CONFIGURATIONS
%-------------------------------------------------------------------------------
% A4 paper size by default, use 'letterpaper' for US letter
\documentclass[11pt, a4paper]{awesome-cv}

% Configure page margins with geometry
\geometry{left=1.4cm, top=.8cm, right=1.4cm, bottom=1.8cm, footskip=.5cm}

% Specify the location of the included fonts
\fontdir[fonts/]

% Color for highlights
% Awesome Colors: awesome-emerald, awesome-skyblue, awesome-red, awesome-pink, awesome-orange
%                 awesome-nephritis, awesome-concrete, awesome-darknight
\colorlet{awesome}{awesome-skyblue}
% Uncomment if you would like to specify your own color
% \definecolor{awesome}{HTML}{CA63A8}

% Colors for text
% Uncomment if you would like to specify your own color
% \definecolor{darktext}{HTML}{414141}
% \definecolor{text}{HTML}{333333}
% \definecolor{graytext}{HTML}{5D5D5D}
% \definecolor{lighttext}{HTML}{999999}

% Set false if you don't want to highlight section with awesome color
\setbool{acvSectionColorHighlight}{true}

% If you would like to change the social information separator from a pipe (|) to something else
\renewcommand{\acvHeaderSocialSep}{\quad\textbar\quad}


%-------------------------------------------------------------------------------
%	PERSONAL INFORMATION
%	Comment any of the lines below if they are not required
%-------------------------------------------------------------------------------
% Available options: circle|rectangle,edge/noedge,left/right
\photo[circle,noedge,left]{./profile}
\name{Nikhil}{Bhasin}
\position{Software Engineer{\enskip\cdotp\enskip}Information Technology Major}

\mobile{(+91)-7428854398}
\email{bhasin.nikhil.12@gmail.com}
\homepage{nikhil-bhasin.netlify.app}
\github{phoenix-1-2}
\linkedin{nikhil1204}
% \gitlab{gitlab-id}
% \stackoverflow{SO-id}{SO-name}
% \twitter{@twit}
% \skype{skype-id}
% \reddit{reddit-id}
% \medium{madium-id}
% \googlescholar{googlescholar-id}{name-to-display}
%% \firstname and \lastname will be used
% \googlescholar{googlescholar-id}{}
% \extrainfo{extra informations}

% \quote{``Be the change that you want to see in the world."}


%-------------------------------------------------------------------------------
%	LETTER INFORMATION
%	All of the below lines must be filled out
%-------------------------------------------------------------------------------
% The company being applied to
\recipient
  {Company Recruitment Team}
  {Google Inc.\\1600 Amphitheatre Parkway\\Mountain View, CA 94043}
% The date on the letter, default is the date of compilation
\letterdate{\today}
% The title of the letter
\lettertitle{Job Application for Software Engineer}
% How the letter is opened
\letteropening{Dear Mr./Ms./Dr. LastName,}
% How the letter is closed
\letterclosing{Sincerely,}
% Any enclosures with the letter
\letterenclosure[Attached]{Curriculum Vitae}


%-------------------------------------------------------------------------------
\begin{document}

% Print the header with above personal informations
% Give optional argument to change alignment(C: center, L: left, R: right)
\makecvheader[R]

% Print the footer with 3 arguments(<left>, <center>, <right>)
% Leave any of these blank if they are not needed
\makecvfooter
  {}
  {Nikhil Bhasin ~~·~~~ Cover Letter}
  {}

% Print the title with above letter informations
\makelettertitle

%-------------------------------------------------------------------------------
%	LETTER CONTENT
%-------------------------------------------------------------------------------
\begin{cvletter}

\lettersection{About Me}
Current Software Development Engineer at a unicorn start-up company, Innovaccer, working in Product Engineering Team to develop the production-ready Backend micro-services. I am an Information Technology graduate from Jaypee Institute of Information Technology, Noida. I worked as a Software Engineering Intern at Innovaccer. I also worked as a Software Engineering Intern at an Investor Relations and Equity Research Company, Internet Bull Report. I am working every day to build, extend, optimize and refactor Backend architecture and its core services.


\lettersection{Why Google?}
There are many reasons I want to join Google. \\
Firstly, I think Google will be making good use of my skill-set. I’ll be working with really great people - people who can get the job done, who act rationally, who don’t make life difficult when it doesn’t have to be.Therefore, I'll enjoy going to work. 
\\
Secondly, I’ve never seen engineering infrastructure as solid as that at Google. I used Google products before joining Google, I continue to use them, most of everybody I know uses them, and they make my life better. Therefore, I feel that I’m making the world a better place by working here.
\\
Lastly, Google pays me enough to support my lifestyle and provide medical insurance, and many other perks. 

\lettersection{Why Me?}
Honestly, I possess all the technical skills and experience that you’re looking for. I’m pretty confident that I am the best candidate for this job role. I am a self-motivated person, and I try to exceed my superior’s expectations with high-quality work. Being a fast learner, I quickly pick up business knowledge related to my project. Lastly, I would like to add that I work well both as an individual contributor and also as a team member. I am praised as adaptable and enthusiastic by my peers. My abilities such as teamwork and critical thinking and problem-solving skills can fill the software engineer role at Google.

\end{cvletter}


%-------------------------------------------------------------------------------
% Print the signature and enclosures with above letter informations
\makeletterclosing

\end{document}
